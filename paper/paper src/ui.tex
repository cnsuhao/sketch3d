%\section{System Overview}
%
%%With the \ca{SKETCH} system, the user only needs to draw sketch over 3D model in a conventional 2D drawing manner: she does not needs to provide additional operation to specify the 3D information of the drawn sketch. Our system automatically bring the input 2D sketch to 3D based on the relation between the drawn strokes and the existing 3D features.
%
%\ca{1. brief introduction of the interface.}  The \ca{SKETCH} system adopts an incremental interface to construct a complete 3D sketch. The user draws the strokes plane by plane, i.e.,
%
%\ca{}

\section{User Interface}

%In this section we describe the interface of our sketching system.
Our prototype is composed of two windows: an input window for sketching and a preview window for displaying candidate 3{D} sketches, which are rendered in a specified view angle such that the user can easily perceive the results. Our user interface is then mainly devised into two interlaced stages: sketching and feedback. In the sketching stage, the user performs sketching on top of a 3{D} model on the screen. In a default mode, our system immediately returns 3{D} strokes for preview. Then, in the feedback stage, the user selectively confirms the current displayed results in 3{D} or override the results if she finds the default results not correct. To keep the sketching process smooth, we provide an intuitive feedback interface so as not to intervene the sketching process, which is necessary only when the current 3{D} content is falsely anchored.
%The operations in the preview window can also help to resolve the ambiguity introduced by the user's input.
%Each time the user draws strokes in the input window, the constructed 3D sketch is shown in the preview window in a carefully-chosen view angle to better express the 3D information of the resulting sketch.
In the following, we describe the interface for sketching and feedback in detail.

\subsection{Sketching}

%\ca{emphasis: our system follows the conventional sketching manner.}

To start with our system, the user first loads a 3{D} model which will be sketched over. This model is displayed in the input window with orthogonal projection, thus the 3D parallel relation is retained in the screen space. The view angle can be manipulated in \ca{the positioning mode} with rotation, translation, and zooming. When a desired view angle is obtained, the user can  enter \ca{the sketching mode} to draw sketch over the model. The drawing is performed using either mouse, digital pen, or a touch device, in a way similar to conventional drawing on paper. To convert the input strokes into 3D, we adopt an incremental strategy to reduce the potential ambiguities which arises with stroke number and let the user to resolve ambiguities when necessary. In particular, during the drawing, each time the user draws a set of strokes, before the feedback stage, we assume all the set of strokes lie on some planar canvas. Then, in a key stage, our system finds the planar canvas automatically by inferring its position and orientation from the relations between the drawn strokes and the model. Each time when the user draws a new stroke, this planar canvas will be updated. When the user finishes a set of strokes, she provides her feedback using the feedback interface (detailed in the Section \ref{sec:feedback}) to confirm the current set of 3{D} strokes and the drawn strokes will be converted into 3{D} by projecting them on the planar canvas. The user then continues to draw another set of strokes or manipulate the view angle. The user can also enter \ca{the erasing mode} to modify the previous drawn 3D strokes (see in the accompanying video).


\subsection{Feedback}\label{sec:feedback}
When the user draws strokes in the input window, our system automatically finds a default planar canvas and converts the drawn strokes into 3{D}. However, the 3{D} information of the sketch can not be perceived in the input window without changing the view angle. Therefore, we display the resulting 3{D} sketches in the preview window in a different view angle. In addition, since the user's input may have ambiguities, the potential planar canvas may not be unique. In this case, our algorithm also finds other possible candidate planar canvases and display the candidate planar canvases with the associated 3{D} sketches to the user in the preview window. The user, can select one from the candidate sketches, if the default one is not desired. To select the desired 3D sketch and associated canvas, the user simply use the stylus or finger to stroke across the desired candidate.  The user can also change the view angle to facilitate the selection process, if necessary. The selection mode and viewing mode can be switched by touching a floating button, which are displayed at the bottom of the preview window.

During the incremental sketching process, besides drawing strokes, an important task for the user to perform is to help confirm the current 3{D} contents and correct falsely estimated 3{D} strokes if exists. We find such interaction intuitive and typical as in traditional drawing system where a user often needs to confirm the current content using some emphatic means (e.g., by drawing heavy strokes over the contours of a character). To accomplish such task, a simple solution could be to include buttons on the task bar, such that the user could click these buttons to issue the desired commands. However, such interfaces often come with an additional travel distance cost as the user needs to move the mouse or finger from one place to another, frustrating the drawing process. Instead, we design a user-friendly interface for this task. We adopt floating buttons to minimize the stylus or finger movement when the user wants to issue a confirmation. During the drawing of a stroke, the buttons are invisible. When the stroke is finished, the floating buttons are shown next to the drawn strokes transparently. If the user continues to draw, the buttons fades away and will not introduce occlusion problem (see Figure \ref{} and the accompanying video for details).






%\ca{automatic view angle estimation}
%The view angle in the preview window is determined by the candidate 3D sketches. We choose a view angle such that the candidates

%\ca{manipulating view angle if necessary}


%\ca{canvas visualization??}

%\ca{selecting desired canvas}
%We have two

%\ca{mode switching by buttons in the bottom: window is small, no need to float}

%The user draws strokes in the input window, and the constructed candidate 3D sketches are shown in the preview window
% with a different view angle in which the difference of all candidate 3D sketches is


